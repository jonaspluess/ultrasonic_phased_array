Ultrasound is commonly used by different technologies to collect information about certain objects. During this process, the echoes of sound waves transmitted by an ultrasonic transducer are measured. Signal processing is then used to gain the aforementioned information.
If an array of ultrasonic transceivers is used (as opposed to just one), then the resulting sound beam will have a stronger directivity. By changing the signal phases of the excitation signals through a delay, the direction of the sound beam can be adjusted.

Since the ultrasonic transceivers used in air have a relatively low resonant frequency of about $30 - 50 \mathrm{kHz}$ , it is possible to build such a phased array system cost-efficiently. For this project an ultrasonic phased array system containing a $1$x$8$ ultrasonic transceiver array has been developed. It scans the surrounding environment in a range of at least $5 \mathrm{m}$ with a depth resolution of about $\pm 5 \mathrm{cm}$. Thanks to a user interface, settings for the measurements can be specified. A range between $\pm 30$ degrees can be scanned automatically. The results are presented as a heatmap-graph representing a picture of the surroundings and a line plot of the newest measured signal. Lastly, an algorithm finds the peaks in the scanned area and visualizes them in the heatmap-graph.

The system contains hardware in the form of a shield for the Arduino DUE and software. The firmware is written as a statemachine in C for the SAM3X8E microcontroller. The measured data is sent to a processing system via USB. This processing system is programmed in python and implemented as a web application. It is possible to run it on a Raspberry Pi. The user interface is accessible through a web browser.
