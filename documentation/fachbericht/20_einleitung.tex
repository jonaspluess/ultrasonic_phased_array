Nicht grundlos ist die Fledermaus mit einem aussergewöhnlichen Sinn zur Orientierung im Raum ausgestattet. Durch die Erzeugung von hochfrequenten Schallwellen und der anschliessenden Auswertung der Echos kann sie Informationen über den Raum gewinnen. Dadurch wird die Fledermaus zum hervorragenden Jäger in der Dunkelheit.
Dasselbe Verfahren machen sich unterschiedliche Technologien zunutze. In der Medizinaltechnologie, der zerstörungsfreien Materialprüfung oder beim Sonar werden so die relevanten Informationen gewonnen. Dabei werden Echos von Ultraschallwellen gemessen und geben Aufschluss über die zu untersuchenden Strukturen.

Wird statt eines einzelnen Ultraschallsenders ein Array aus mehreren Sendern verwendet, kann eine stärkere Richtwirkung des Schallkegels erzielt werden. Mithilfe von Zeitverzögerungen (Phasenverschiebungen) zwischen der Ansteuerung der einzelnen Sender kann die Richtung dieses Kegels in einen gewünschten Winkel gelenkt werden. Ein solches System wird als Ultraschall Phased Array bezeichnet.

Nicht nur in Flüssigkeiten oder Festkörpern, sondern auch in Luft kann ein solches Ultraschall Phased Array verwendet werden, um Informationen über die Umgebung zu gewinnen. Die für diesen Anwendungsfall benötigten Ultraschalltransceiver und deren typischerweise niedrige Eigenfrequenz im Bereich von $30 - 50 \mathrm{kHz}$ ermöglichen eine kostengünstige Umsetzung. Im Rahmen dieser Bachelor-Thesis wird ein solches Ultraschall Phased Array in Luft entwickelt, welches es ermöglicht, über den Winkel und die Distanz Objekte im Raum zu lokalisieren.

Im vorhergehenden Projekt 5 wurden theoretische Grundlagen als Basis für dieses Projekt erarbeitet und ein erster Prototyp angefertigt. Die Hardware des letzten Projektes besteht aus einem PCB-Shield für das Arduino DUE Board, welches über ein getrenntes $1$x$8$ Sende- und $1$x$8$ Empfangsarray verfügt. Gesteuert wird dieses vom SAM3X8E-Mikrocontroller auf dem Arduino DUE. Empfangene Ultraschallsignale werden über eine USB-Schnittstelle an ein weiterverarbeitendes Desktop-Betriebssystem gesendet.

Dieses System wird im Rahmen dieser Bachelor-Thesis weiterentwickelt. Um die verwendete Anzahl Ultraschalltransceiver zu verkleinern und damit die Kosten zu senken, wird die Elektronik so entwickelt, dass jeder Ultraschalltransceiver im Array sowohl als Sender als auch als Empfänger verwendet wird. Um die minimale Messdistanz zu verkleinern, können die einzelnen Ultraschalltransceiver nach dem Sendevorgang gedämpft werden. Nebenkeulen in der Richtcharakteristik können wahlweise durch verschiedene Amplitudenbelegungen unterdrückt werden. Die Firmware nimmt per USB-Schnittstelle Steuerbefehle entgegen und führt anhand dieser einen Messvorgang aus. Die Software für das Desktop-Betriebssystem ist als Webapplikation implementiert, welche hostseitig die Signalverarbeitung übernimmt und clientseitig ein User-Interface zur Verfügung stellt.

Über das User-Interface kann ein automatisierter Scan der Umgebung für einen gewünschten Winkelbereich zwischen $\pm 30^{\circ}$ durchgeführt werden. Resultate werden parallel zum Scanvorgang als aktuelles Bild der Umgebung und als Linienplot der neusten Messung dargestellt. Ein Algorithmus durchsucht die eintreffenden Daten und berechnet daraus die Positionen von Maxima innerhalb des gescannten Winkelbereichs, welche den Positionen von reflektierenden Objekten im Raum entsprechen.

Die folgende technische Dokumentation beginnt mit dem Grobkonzept und den theoretischen Grundlagen, welche teilweise aus dem letzten Projekt übernommen wurden. Aufgeteilt in Hardware und Software wird daraufhin die detaillierte Umsetzung des Systems beschrieben. Ab\-schliessend werden das Testkonzept und die erreichten Ergebnisse dokumentiert.
